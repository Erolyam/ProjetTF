\documentclass[10pt,a4paper]{article}
\usepackage[utf8]{inputenc}
\usepackage{amsmath}
\usepackage{amsfonts}
\usepackage{amssymb}
\usepackage{graphicx}

\author{
	Ibrahim \textsc{Akrach}
	\\
	Quentin \textsc{Guillien}
	\\
	Sammy \textsc{Loudiyi}
	\\
	Yiwei 	\textsc{Pang}
	\\
	Diego \textsc{Romero Rodriguez}}

\title{\textbf{Projet de Test Fonctionnel} \\ Compte-rendu du sprint 1}




\begin{document}
	\maketitle

	\newpage

	\tableofcontents	
	
	\section{Compte-rendu du planning poker initial}
	
	\subsection{Quelles techniques pour découper en fiches ?}
	\subsection{Quelles techniques pour coter les fiches ?}
	\subsection{Quelles stratégies de planification pour le Sprint ?
	}
	\subsection{Quelles difficultés rencontrées / solutions apportées ?
	}
	
	\section{Rétrospective vis-à-vis des exigences client}
	\subsection{Quelle couverture des exigences / tests d'acceptation ?}
	\subsection{Quelles difficultés rencontrées / solutions apportées ?}

	
	\section{Organisation de l'équipe d'un point de vue RH}
	\subsection{Qui développe ?}
	\subsection{Qui teste ?}
	\subsection{Qui pilote les équipes ?}
	\subsection{Quelles organisations pour affecter les fiches aux binômes ?}
	\subsection{Quelles pratiques pour synchroniser les différents binômes ?}
	
	\section{Organisation de l’équipe d’un point de vue technique}
	\subsection{Quels environnements de développement ?}
	\subsection{Quels outils de tests ? Pour quelle couverture ?}
	\subsection{Quelle plateforme d’intégration ?}
	
	\section{Bilan et prise de décision d’amélioration continue pour le Sprint suivant }
	\subsection{Proposer un proverbe ou dicton pour caractériser le travail de l’équipe lors de ce premier Sprint (à présenter sous une forme individuelle). Expliquez ce choix.}
	\subsection{Quel sentiment collectif sur le processus de développement mis en place ? Ses atouts ? Ses freins ?}
	\subsection{Quels ajustements en fonction de l’avancée réelle par rapport aux prévisions ?}
	\subsection{Quelles décisions pour améliorer les difficultés relevées (techniques et/ou organisationnelles) ?}
	
\end{document}


